
Protein folding remains one of the outstanding challenges in
structural biology \cite{dill2012folding}. While it is usually defined
as the problem of predicting the three-dimensional (3D) structure of a
protein from its amino acid sequence, it can also be cast as a ranking
problem: Given a number of structural models of various quality, can
we computationally predict how close each of these models is to the
native fold of the protein?

Progress in the field is monitored through the Critical Assessment of
protein Structure Prediction (CASP) competition \cite{moult1995large},
in which protein folding methods are evaluated in terms of their
accuracy at predicting protein structures ahead of their
publication. Most methods in competition include a conformational
sampling step, which generates a number of plausible protein
conformations, and a quality assessment step, which attempts to select
those conformations closest to the unknown native structure.

In this paper we explore the application of deep learning to the
problem of ``model quality assessment'' (MQA), also called
``estimation of model accuracy'' (EMA) \cite{kryshtafovych2015}. Deep
learning has recently garnered considerable interest in the research
community \cite{lecun2015deep}, particularly in computer vision and
natural language processing. Unlike more ``shallow'' machine learning
approaches, deep learning improves performance by learning a
hierarchical representation of the raw data at hand. It alleviates the
need for feature engineering, which has traditionally constituted the
bulk of the work done by researchers.

Deep learning has recently been applied to biological data and has
yielded remarkable results for predicting the effects of genetic
variations on human RNA splicing \cite{xiong2015human}, for
identifying DNA- and RNA-binding
motifs \cite{alipanahi2015predicting}, and for predicting the effects
of non-coding DNA variants with single nucleotide
precision \cite{zhou2015predicting}. These successes have one thing in
common: they use raw data directly as input and do not attempt to
engineer features from them.

\subsection{Related work}

Deep learning methods have also been used for protein structure
quality assessment \cite{nguyen2014dlpro, cao2016deepqa,
uziela2017proq3d}. For instance, DeepQA \cite{cao2016deepqa} uses 9
scores from other MQA methods and 7 physico-chemical features
extracted from the structure as input features to a deep Boltzmann
machine \cite{}. The method has been reported \cite{cao2016deepqa} to
outperform ProQ2 \cite{ray2012proq2}, which was the top-performing
method in the CASP11 competition \cite{kryshtafovych2015}.
ProQ3D \cite{uziela2017proq3d} uses the same high-level input features
as the earlier ProQ3 method \cite{uziela2016proq3} (including the
Rosetta \cite{leaverfay2011rosetta} score), but achieves better
performance by replacing the support vector machine by a deep neural
network. Since the original ProQ3 method had one of the top
performances in CASP12 \cite{elofsson2017qacasp12}, it can be expected
that ProQ3D performs equally well. Although both DeepQA and ProQ3D
methods are based on deep neural networks, they use high-level
features as input. In that sense, they use DL models more as
traditional ``shallow'' classifiers than as end-to-end learning
models. It is likely that they do not get all the advantages offered
by the DL approach.
%
By comparison, the DL-Pro algorithm \cite{nguyen2014dlpro} uses a
sligthly more raw input, consisting of the eigenvectors resulting from
a principal component analysis of the C$\alpha$-to-C$\alpha$ distance
matrix. The model itself is an autoencoder \cite{} trained to classify
the structures into either ``near native'' or ``not near native''.
%%% How is that model doing compared to the others?

More in line with the ``end-to-end training'' spirit of deep learning,
methods using as input a 3D representation of the structure have been
developed to score protein-ligand poses \cite{wallach2015atomnet,
ragoza2017ligandscoring}, to predict ligand-binding protein
pockets \cite{jimenez2017deepsite}, or to predict the effect of a
protein mutation \cite{torng2017}. All these
methods\cite{wallach2015atomnet, ragoza2017ligandscoring,
jimenez2017deepsite, torng2017} treat the molecules of interest as 3D
objects represented on a grid. While a rigorous comparison of each of
these methods with is not always possible, they generally appear to
improve on the state of the art: Both
AtomNet \cite{wallach2015atomnet} and the 3D convolutional neural
network of Ragoza et al. \cite{ragoza2017ligandscoring} perform
consistently better than either Smina \cite{koes2013smina} or AutoDock
Vina \cite{trott2009vina}.
%
For small molecules, Sch\"{u}tt and coworkers \cite{schutt2017nc,
schutt2017moleculenet} have developed deep neural networks to predict
the molecular energy of a variety of chemical compounds in various
conformations (or even various isomeric states). These models,
intended to be used as universal force fields, are trained on ab
initio energies and forces, and use only the nuclear charges and the
interatomic distance matrix as input.
