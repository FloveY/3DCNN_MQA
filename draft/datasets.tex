
We train and assess our method using the protein decoy datasets from
the CASP competition \cite{moult2014critical}.  We use the CASP7 to
CASP10 data as training set and the CASP11 data as test set, for a
total of 564 target structures in the training set and 83 target
structures in the test set. Each target from the training set has 282
decoys on average.
%
The test dataset is split into two subsets \cite{kryshtafovych2015}:
\tchanged{``stage 1'' with 20 decoys per target selected randomly from
all server predictions, and ``stage 2'' with 150 decoys per target considered best by the
Davis-QAconsensus evaluation method \cite{kryshtafovych2015}.}
%
The native structures were not included in the analysis, neither
during the training phase nor during the testing phase. To make the
structural data more consistent, the side chains of all decoy
structures were optimized using the SCWRL4 program
\cite{krivov2009improved}.

\tchanged{The distributions of target sequences lengths 
cover the same range of sequence lengths (SI Fig. \ref{Fig:dataLengthDist})}. 
To ensure that the training and test sets are significantly different, we have aligned all test
sequences against all training sequences using blastp \cite{altschul1990basic}. 
%The most significant alignments (E-value${}
%< 10^{-4}$) are shown in Table \ref{Tbl:datasetsSimilarity}.  
Less than 11\% of the targets in the test set (9 out of 83) have sequence
similarity with any target in the training set (SI Table \ref{Tbl:datasetsSimilarity}).

To further assess the evolutionary similarity of the two datasets, we
have computed their overlap in terms of Pfam families
\cite{finn2016pfam}. Pfam families were found using \tchanged{HMMER \cite{finn2015hmmer}}
with an E-value cutoff of 1.0 \cite{finn2016pfam}.  
%Table \ref{Tbl:SharedPfam} shows the targets in the test and train sets that
%share a family. 
Accounting for targets for which no Pfam family could
be determined, approximately 25\% of the test set targets share a
family with approximately 10\% of the training set targets (SI Table \ref{Tbl:SharedPfam}).

We have also compared the structures in the training and test sets
using the ECOD database \cite{cheng2014ecod}. This database provides a
five-level classification of all structures of the RCSB PDB
\cite{berman2000protein} according to the following criteria:
architecture (A-group), possible homology (X-group), homology
(H-group), topology (T-group), and family (F-group).  Since the ECOD
classification is domain-based, multi-domain protein chains can belong
to multiple A-, X-, H-, T-, or F-groups.  The higher the level two
protein domains occupy, the more structurally similar they are.
%
\tchanged{The summary of the overlap} between the training and test sets
is presented in Fig. \ref{Fig:summaryTable}. For each target domain in
the test set (T0759 to T0858), a black tile indicates that at least
one structure from the training set belong to the same ECOD group (A,
X, H, T, and F). 
%
\tchanged{The more detailed information on the classification and 
overlap of the training and test sets can be found in the Supplementary Information.}

\begin{figure}[H]
    \makebox[\textwidth]{
    \includegraphics[width=\paperwidth]{Fig/summary_table_cmyk.eps}
    }
%
    \caption{Overlap of the training set on each target domain of the
    test set (from T0759 to T0858). The first 5 rows of tiles
    correspond to the ECOD classification of protein domains (A-, X-,
    H-, T-, and F-groups). A black tile in any of these rows indicates
    that at least one structure from the training set belongs to the
    same ECOD group as the target. Targets for which no ECOD
    classification is available are left empty.
%%% I see that all targets excluded from the analysis have an empty
%%% row of squares. Is T0838 excluded as well? What about the targets
%%% that are not in the list? (775, 778, 779, 791, 793, 795, 799, 802,
%%% 804, 809, 826, 828, 839, 842, 844, 846, 850) Were they all
%%% excluded from the CASP competition? The CASP11 QA paper mentions
%%% that the following targets were cancelled by the organizers: 778,
%%% 779, 791, 809, 842, 844, 846, 850. What about the other ones?
    A black tile in the ``Family'' row indicates that at least one
    structure from the training set belongs to the same Pfam family as
    the target. (A grey tile indicates that no Pfam family information
    is available for the target.) The ``Clan'' row shows similar
    information for Pfam clans. A black tile in the ``Alignment'' row
    indicates that at least one sequence in the training set aligns to
    the target sequence with an E-value smaller than $10^{-4}$.}
%
    \label{Fig:summaryTable}
\end{figure}
