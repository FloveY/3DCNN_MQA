
In this work we showed that it is possible to construct an algorithm that learns to assess the quality of protein models from 
raw representation of the model. Our work used the atom types densities as such representation, however it is clear that 
any other physical quantity defined on the grid can be employed. The examples of such quantites are electrodymanic potential and 
water oxygen density distribution. So far, no other quality assessment method was able to include these 
crucial properties of the solute.

This work also identified some important issues: only approximate score invariance under transformations and the difficulty of 
interpretation of the results. The invariance problem can be solved using the recently published approach by Worral et al. \cite{worrall2016harmonic}.
They restricted the space of coefficient of the convolutional filters to the circular harmonics to attain eqivariance of transformations at each layer 
of the network under rotations. The layerwise equivariance then leads to the invariance of the final output.

The interpretation difficulty of deep neural networks is still an important research problem in machine learning. However, the recently 
published approach to quantify interpretability \cite{bau2017network} of deep learning models indicates the rapid advancements in this field. 
The authors of this approach used the exhaustively labeled image dataset, that contains the bounding boxes and labels for 
fine-grained features, such as body parts or car parts. In the case of protein models such labels are readily available: 
the secondary structure elements, amino-acids, hydrogen bonding network and disulfide bonds, as well as others.

Alltogether, we believe that this work opens the way to directly use the information about protein environment, aleviates the need to 
engineer features and shows that the drawbacks of this approach can be solved with the further research. To accelerate the 
mentioned process we publish the code as well as the preprocessed datasets along with this paper \cite{AnonimousLink}.
